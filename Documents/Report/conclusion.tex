%%%%%%%%%%%%%%%%%%%%%%%%%%%%%%%%%%%%%%%%%%%%%%%%%%%%%%
\section{Conclusion}
\label{sec:conclusion}

Iroko represents a first attempt at using machine learning
to predict data center traffic. The endeavour relies crucially
on data center traffic having enough structure to in fact allow
some level of prediction. Yet even for largely random traffic
Iroko showed decent results, often besting baseline solutions.
Iroko is only the opening salvo, but the attempt has merit.Iroko
struggles with reducing drop rate, and some random traffic
patterns. With more fine-tuned
prediction, and better rate-limiting algorithms, Iroko can likely
overcome these challenges. 
More work must be done before claims of superiority to all
incumbent solutions can be made, but there are reasons to be
optimistic. 

Machine learning is perhaps the king of "black-box" solutions, 
and it's efficacy is undeniable if sometimes misleading. While
the standoff between so called "white-box" and "black-box" is 
likely to continue for the foreseeable future, Iroko shows
that "black-box" solutions do have a role to play. It may turn
out that neither comes to rule, but rather future SDN's will 
employ both techniques to build ever more efficient solutions.
It may well turn out that the future is grey.



