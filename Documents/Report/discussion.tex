%%%%%%%%%%%%%%%%%%%%%%%%%%%%%%%%%%%%%%%%%%%%%%%%%%%%%%
\section{Discussion}
\label{sec:discussion}

Not unexpectedly, Iroko struggles on a few random traffic patterns, however we
do not see this as particularly concerning. Firstly Iroko still outperforms
both DCTCP and ECMP implementations on most random traffic patterns, and in general.
Further, as these traffic patterns
were designed by the authors of the Hedera paper \cite{hedera}, it remains
likely these patterns are somewhat amenable to the Hedera implementation.
Finally as discussed, fully random traffic patterns seem unlikely to be prevalent in a
datacenter context. While the general predictability of datacenter
traffic remains an area of open research, our results show that
if there is sufficient predictability, we do get results better than
random, and for truly random data, Iroko does no worse generally than
incumbent solutions. 

Our machine learning approach shows promise, but machine learning remains
largely a black-box discipline. This nature makes it difficult for us
to determine precisely which parameters to use, and what drives
the improvements we've seen. Our current machine learning approach
is remarkably light-weight and does not require vast resources
in our test network. A heavier handed approach may yield better
results, however these may come at the cost of scalability. We
also noted that further tuning of the current implementation is
likely to lead to stronger results. Due to testing limitations, 
the controller had a very short number of periods to learn 
on. In longer running cases, we full expect Iroko to 
deliver stronger results as there is more time for machine learning 
algorithms to converge. 

For efficient routing, we used EMCP with Iroko, however better
routing regimens exist, in particular Hedera. While we compared
Iroko to Hedera, in fact the two systems are largely orthogonal
and may in fact benefit from one another. Iroko does not directly deal
with routing or so called elephant flows. Thus Iroko could be
further improved by employing a routing scheme similar to that of
Hedera, while still retaining bandwidth prediction capabilities.


