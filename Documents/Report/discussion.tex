%%%%%%%%%%%%%%%%%%%%%%%%%%%%%%%%%%%%%%%%%%%%%%%%%%%%%%
\section{Discussion}
\label{sec:discussion}

Rather predictably, Iroko struggles on random traffic patterns, however we
do not see this as particularly concerning. Firstly \textit{Iroko} still outperforms
both DCTCP and ECMP implementations. Further, as these traffic patterns
were designed by the authors of the Hedera paper \cite{hedera}, it remains
likely these patterns are somewhat amenable to the Hedera implementation.
Finally random traffic patterns seem unlikely to be prevalent in a
datacenter context. While the general predictability of datacenter
traffic remains an area of open research, our results show that
if there is sufficient predictability, we do get results better than
random, and for truly random data, Iroko does no worse generally than
incumbent solutions. 

Our machine learning approach shows promise, but machine learning remains
largely a black-box discipline. This nature makes it difficult for us
to determine precisely which parameters to use, and what drives
the improvements we've seen. Our current machine learning approach
is remarkably light-weight and does not require vast resources
in our test network. A heavier handed approach may yield better
results, however these may come at the cost of scalability. 
