%%%%%%%%%%%%%%%%%%%%%%%%%%%%%%%%%%%%%%%%%%%%%%%%%%%%%%
\section{Future Work}
\label{sec:future}

Predicting datacenter traffic online with machine learning shows promise, and we believe that 
future widely deployed datacenter traffic control mechanisms will include
machine learning in their toolkit. However machine learning alone may not
be sufficient to achieve the best possible results. While we  
compared our results to Hedera, the work remains largely
orthogonal to our approach, and scheduling flows would likely 
complement machine learning based bandwidth control. 

For large, long-running workloads that require significant 
network resources at the termination such as MAP-REDUCE
workloads, host may have very predictable completion times
for these jobs. In such cases, hosts could report these 
estimated completion times to the central controller,
allowing the controller to account for them pre-emptively
rather than trying to infer them which may take a couple
of epochs, and therefore reduce efficiency. Such a
system would be complementary to \textit{Iroko} and
allow for finer grained control.

Finally employing a token or credit based system similar
to that explored in \cite{expresspass} might yield excellent
results when coupled with our machine learning based predictive
central controller. This would allow hosts more ability to manage
their own bandwidth with intertemporal substitution of flows,
while still enforcing centralized predictive control.

