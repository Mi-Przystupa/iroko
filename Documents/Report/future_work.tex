%%%%%%%%%%%%%%%%%%%%%%%%%%%%%%%%%%%%%%%%%%%%%%%%%%%%%%
\section{Future Work}
\label{sec:future}

Predicting datacenter traffic online with machine learning shows promise, and we believe that 
future widely deployed datacenter traffic control mechanisms will include
machine learning in their toolkit. However, machine learning alone may not
be sufficient to achieve the best possible results. Machine learning
based central traffic controllers can benefit from dynamic flow
scheduling as well as further traffic engineering. Thus expanding
the Iroko framework to include flow rather than interface based 
predictions, or a combination of both, would be an interesting 
avenue to pursue. Combined with a robust routing algorithm, we
believe such a solution could show superior performance when
compared to virtually all currently existing solutions.

For large, long-running workloads that require significant 
network resources at the termination such as MAP-REDUCE
workloads, hosts may have very predictable completion times
for these jobs. In such cases, hosts could report these 
estimated completion times to the central controller,
allowing the controller to account for them pre-emptively
rather than trying to infer them, which may take a couple
of epochs, and therefore reduce efficiency. Such a
system would be complementary to Iroko and
allow for finer grained control. By taking further
host level input, Iroko could reason about individual 
flows, and allocate not only bandwidth, but also 
specific paths.

Finally employing a token or credit based system similar
to that explored in \cite{expresspass} might yield excellent
results when coupled with our machine learning based predictive
central controller. This would allow hosts more ability to manage
their own bandwidth with inter-temporal substitution of flows,
while still enforcing centralized predictive control.

