%%%%%%%%%%%%%%%%%%%%%%%%%%%%%%%%%%%%%%%%%%%%%%%%%%%%%%
\section{Motivation}
\label{sec:motivation}
Several recent contributions in networking research 

\paragraph{Software-Defined Networking}
With the advent of Software-Defined Networking (SDN), operators now have the 
ability to freely control and adapt their network architecture, leading to 
highly customized systems and fine-grained optimization.~\cite{sdn_road}

Moving away from the principle of distributed communication and routing, SDN 
introduces the notion of "centralized management". A single controller with 
global knowledge is able to automatically modify and adapt the forwarding 
tables of all switches in the network, while notifying end hosts of changes in 
the network.
This full architectural control and centralized management facilitated new 
opportunities in the space of TCP congestion research. Traffic can now be 
managed in a  \textit{centralized} fashion based on \textit{global knowledge} 
of the entire topology and traffic patterns.

\paragraph{Centralized Scheduling and Congestion Control}
A new line of centralized schedulers has emerged that can achieve close to 
optimal bandwidth utilization.~\cite{hedera, fastpass, microte, b4, dionysus}
However, these schedulers are still  \textit{reactive}  in nature. The central 
controller responds to changes in the network or requests by applications, 
which may cost valuable round-trip latency. Often, short-term flows or bursts 
are unaccounted for, which causes undesirable packet loss and back-propagating 
congestion.

\paragraph{Admission-control systems}
In parallel, admission-control-based approaches have gained traction as a 
viable alternative to the conventional burst-and-backoff mechanism of 
TCP.~\cite{expresspass, fastpass, perc}
The idea of admission control and service guarantees in networks is not 
new.~\cite{access_limit, access_limit2}. However, such designs traditionally 
aim to assure quality and bandwidth guarantees in a contentious, decentralized, 
and untrusted environments such as the internet. In a datacenter, these 
conditions do not apply. End-hosts are generally considered reliable and 
restricted in behaviour, which allows for great simplification of enforcement 
and prioritization policies.

\paragraph{Predictability of traffic}
