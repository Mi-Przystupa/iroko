%%%%%%%%%%%%%%%%%%%%%%%%%%%%%%%%%%%%%%%%%%%%%%%%%%%%%%
\section{Related Work}
\label{sec:related}
Network traffic congestion control and fairness has been an active topic of 
research for several decades and exhibits a vast range of work. We draw the 
majority of our inspiration from the following four categories of research.
\subsection{End-to-End TCP optimization}
Many different TCP optimizations tailored to various environments and use-cases 
exist. In general, these extensions share TCP's fundamental requirements of 
scalability, decentralization, and fairness. Optimizations are generally 
static, only on the end-host, and implemented "offline".~\cite{throwdown} This 
family of modifications includes tweaks to the congestion window size and 
growth, signal detection, and optimization goal(e.g., bandwidth, latency, or 
convergence).~\cite{tcp_family}
In this project we compare to two particular variants of the traditional TCP 
protocol. Data Center TCP~\cite{dctcp}  

\subsection{Centralized Scheduling}
SDN
\subsection{Network Admission-Control}
\subsection{Predicting and Learning Network Traffic}
Winstein and Balakrishnan,~2013\cite{remy} demonstrated that it is possible to 
learn optimal TCP congestion control parameters for various network 
topologies. 
Benson et al.,~2014\cite{microte} analyzed data center traffic and inferred 
that around 30\% 
