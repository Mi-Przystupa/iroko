%%%%%%%%%%%%%%%%%%%%%%%%%%%%%%%%%%%%%%%%%%%%%%%%%%%%%%
\section{Related Work}
\label{sec:related}
Network traffic congestion control and fairness has been an active topic of 
research for several decades and exhibits a vast range of work. We draw the 
majority of our inspiration from the following four categories of research.

\subsection{End-to-End TCP optimization}
Many different TCP optimizations tailored to various environments and use-cases 
exist. In general, these extensions share TCP's fundamental requirements of 
scalability, decentralization, and fairness. Optimizations are generally only 
implemented on the end-host and the model itself is static and 
"offline".~\cite{throwdown} This family of modifications includes tweaks to the 
congestion window size and growth, signal detection, and optimization goal 
(e.g., bandwidth, latency, or convergence).~\cite{tcp_family}

In this project, we compare Iroko to two particular TCP protocols of interest.  
The first, Data Center TCP (DCTCP)~\cite{dctcp}, is a congestion control 
algorithm developed to reduce bufferbloat in datacenter networks. DCTCP 
leverages Explicit Congestion Notifications (ECNs) as early warning signals to 
identify queue buildup and adjusts its sending rate to the amount of ECN-marked 
packets on the receiving side. As part of the static TCP congestion control 
family, DCTCP is fully decentralized, does not limit host sending rates in 
advance and relies on a static congestion control model.

The second algorithm we compare is TCP CUBIC~\cite{cubic}, the default 
congestion control protocol of the Linux kernel. CUBIC uses a cubic function   
to increase window size and serves as the baseline for Iroko. It is intended to 
work well over high bandwidth, high latency networks.

\subsection{Centralized Scheduling}
Hedera~\cite{hedera} is an early example of a centralized scheduler working in 
synergy with an Equal-Cost-Multipath (ECMP) routing environment. Hedera 
monitors switch traffic and identifies long-running elephant flows which may 
collide with short-lived, bursty traffic. Hedera overrides the ECMP routing 
entry and moves the elephant flow to an underutilized link, thus reducing the 
potential for flow collision and congestion.

MicroTE~\cite{microte} is a central scheduler working under the 
assumption that a subset of short-term traffic is predictable. The controller 
isolates flows which it determined to be predictable and installs a custom 
optimal routing entry. The remaining unpredictable traffic is routed according 
to ECMP striping.

Hedera and MicroTE rely on TCP to converge to optimal sending rates and do not 
perform any global sending rate optimizations.
Iroko as a rate-limiting arbiter is not intended to compete with centralized 
flow schedulers and may in fact benefit from building on existing routing 
frameworks.

\subsection{Network Admission-Control}
Recent contributions have shown that purely admission-based approaches 
may work as well as, if not better than, conventional unbounded TCP. A large 
focus of these systems is to eliminate queuing and fast flow 
convergence coupled with minimal round trip latency.

FastPass~\cite{fastpass} employs an arbiter mechanism, in which hosts request 
access to the network resources. For any new flow, the FastPass shim layer will 
send a "admission request" to a central controller, which chooses an optimal 
allocated time slot and path. Once an applications has been assigned this path, 
it is guaranteed to experience zero queuing latency and jitter during 
transmission. This "zero-loss" network provides applications with a stable and 
reliable transmission experience. However, FastPass still works reactively and 
only operates on incoming requests. Queueing is thus pushed to end-hosts and 
the arbiter itself.

Another recent example of admission control is the 
ExpressPass~\cite{expresspass} system. As opposed to FastPass, ExpressPass is 
fully distributed with the explicit intention to scale. ExpressPass uses a 
token system enforced by switches in the network. Receiving hosts provide 
senders with tokens as currency, allowing them to send traffic. If the network 
experiences congestion, switches drop the tokens given by the receiver, thus 
limiting the amount of packets a host can send. In ExpressPass, congestion 
control is managed by the network elements instead of end-hosts.

Iroko builds heavily on the contributions of both FastPass and ExpressPass. 
However, we intend to go further and schedule rate-limits ahead of time in 
order to substantially reduce sending delays and host starvation of admission-control systems.



\subsection{Predicting and Learning Network Traffic}
A new direction of networking research is study on the predictability and 
trainability of network traffic.~\cite{throwdown, learning_tcp}
A common assumption is that datacenter traffic is largely unpredictable and 
untrainable.~\cite{fb_dc}
However, Winstein and Balakrishnan,~2013 \cite{remy} demonstrated that it is 
possible to learn near-optimal TCP congestion control parameters for various 
network topologies. The system, Remy, represents an offline learning algorithm 
inferring optimal TCP parameters based on pre-supplied topologies and 
traffic-models.
While Remy only performed static offline training, PCC~\cite{pcc,throwdown} 
aims to develop a dynamic and actively improving congestion control algorithm. 
PCC makes no prior assumption about the network and continuously reiterates over
its host-local optimization function to find optimal parameters.

As part of the contribution of MicroTE, Benson et 
al.,~2014~\cite{microte} analyzed datacenter traffic and 
inferred that around 30\% of short-lived traffic is predictable over a period 
of 1-2 seconds.

In this work, we assume that traffic may be trainable to a degree that is 
useful. On the basis of the aforementioned work we conjecture that it may be 
possible to identify recurring patterns in a subset of traffic and adjust flow 
and congestion control pre-emptively.
