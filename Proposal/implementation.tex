%%%%%%%%%%%%%%%%%%%%%%%%%%%%%%%%%%%%%%%%%%%%%%%%%%%%%%
\section{Implementation}
\label{sec:implementation}
We intend to emulate our system in Mininet~\cite{mininet} to observe traffic patterns and infer a suitable token algorithm. Mininet has proven itself to be a viable tool to model new congestion control algorithms~\cite{mininet_learning}, and will help us prototype our concept efficiently. We will build a custom SDN controller that interacts with traditional OpenFlow software switches as well as end-hosts. End-hosts will run a custom real-world traffic generation script which adjusts based on information packets sent by the controller. 
If time permits, we may expand our implementation to MaxiNet, which can emulate large-scale network stress tests of thousands of nodes on multiple physical hosts.

We initially considered a second implementation alternative in C/C++ based on the FastPass~\cite{fastpass} source code. This would provide us with a fully deployable system which we could fork our implementation from. A major advantage of this approach is the ability to test scenarios and traffic algorithms using real software code. 
However, several concerns made us favour a Mininet emulation instead.
Firstly, FastPass relies on DPDK integration, which requires actual hardware interfaces. The central arbiter in the FastPass design would need to run on a dedicated machine, which increases prototyping and development complexity substantially.
Secondly, the FastPass code is highly specialized and optimized research code with only little available documentation. Modifying and evolving the source code will require thorough understanding of kernel and networking development, a significant time-sink. For a class project, these may be major initial hurdles, taking away from the research aspect of the design concept.
Consequently, we have decided to pursue an approach which allows us to quickly develop an understanding of the problem without being obstructed by engineering work.

