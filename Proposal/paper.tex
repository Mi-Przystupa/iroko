\documentclass[sigconf]{acmart}

\usepackage{booktabs} % For formal tables
\usepackage[normalem]{ulem}
% Copyright
\setcopyright{none}

\input{annotations}

\begin{document}

\title{TCPTokens: Introducing Currency into Data Center Congestion Control}

\author{Anand Jayarajan}
\affiliation{%
  \institution{University of British Columbia}
}
\email{anandj@cs.ubc.ca}

\author{Michael Przystupa}
\affiliation{%
  \institution{University of British Columbia}
}
\email{michael.przystupa@gmail.com}

\author{Robert Reiss}
\affiliation{%
  \institution{University of British Columbia}
}
\email{rreiss@cs.ubc.ca}

\author{Fabian Ruffy}
\affiliation{%
  \institution{University of British Columbia}
}
\email{fruffy@cs.ubc.ca}

\begin{abstract}
Datacenter networks have become a hotbed for research in the recent past. Many works have been published on improving the bisection bandwidth, reducing cost, and improving congestion control as well as latency. As a core contribution of this research, centralized scheduling has entered the stage as dominant strategy to manage flows. Benefitting from a global view and full control over the network, centralized schedulers are able to enforce fine-grained traffic control, frequently achieving near-optimal bandwidth optimization. However, modern congestion control algorithms and schedulers are still fundamentally reactive. Many algorithms are designed to only respond to packet loss or significant increase in latency, not to actively prevent it.

In this project, we plan to explore the opportunities of employing a preemptive and tightly controlling central network scheduler. Using tokens as a global tool to enforce traffic limiting and policies, the scheduler is able to proactively control the flow of traffic as well as respond to application requests. "Knowledge" and predictive analysis in networks is a growing trend in research which we intend to leverage in this system.

We investigate the potential in such a new form of interactive congestion control and analyze it against to state-of-the-art solutions. Our tool of choice to model our system is Mininet, a rapid prototyping emulator for datacenter networks. We will compare our "TCPToken" system against established TCP-congestion systems such as Hedera and DCTCP. Based on the findings and measurement results, we will reassess the feasibility and prospect of a token-based, predictive congestion control algorithm.
\end{abstract}



\keywords{TCP, Congestion Control, SDN, Data center}


\maketitle

%%%%%%%%%%%%%%%%%%%%%%%%%%%%%%%%%%%%%%%%%%%%%%%%%%%%%%%
\section{Introduction}
\label{sec:intro}

Initially, research in network congestion  was dominated by the  assuming a decentralised, autonomous network. End-hosts only have control over the amount of traffic they send, and are unaware of the intentions or traffic rates of their peers. Similarly, switches and routers are unaware of global traffic patterns and only forward based on their local notion of optimality.

In line with these assumptions, TCP was designed to optimize traffic globally on a simple fairness principle; nodes react to packet loss and assume that others will behave accordingly. An equilibrium is reached when all end-hosts achieve a traffic rate that, in sum, conforms to the maximum available bandwidth of the most congested link.
TCP works well  in scenarios where many different distrustful participants compete for limited bandwidth. Still, TCP is a \textit{reactive protocol}; the fact that packet loss and latency increases occur in the network already indicates a problem. Packet loss is primarily  due to overflowing  queues in forwarding elements, implying that traffic has not been optimally distributed. 
Ideally, a network should always be “zero-queue”, i.e., latency will merely be induced by propagation, and not queuing delay. 

Queueing has generally not been a dominating issue in wide-area and enterprise networks, as traffic is sufficiently distributed and diverse, with only few “hot” target hosts.~\cite{hedera, microte} Traffic optimization is a substantial challenge; network operators have no control over the individual network elements nor its participants. Under these conditions, TCP and its extensions can be considered a best-effort solution.

Developments in the past decade have changed the general networking environment. Datacenters have emerged as an exciting new research frontier, posing novel  design challenges and opportunities.
Driven by minimization of costs and maximization of compute power;  data centers must  run at maximum utilization to achieve an optimal compute/cost ratio. Inefficient routing can quickly lead to bufferbloat~\cite{bufferbloat} and the eventual collapse of a high-load network, requiring more sophisticated approaches to solve congestion control. 
On the other hand, operators now have the ability to freely control and adapt their network architecture leading highly customized systems and fine-grained optimization. As a result  Software-Defined Networking (SDN) emerged as a new networking paradigm. Moving away from the principle of distributed communication and routing, SDN introduces the notion of “centralized management”. A single controller with global knowledge is able to automatically modify and adapt the forwarding tables of all switches in the network, as well as notify end hosts of changes in the network.
These two new trends in systems facilitated impactful new innovation opportunities in the space of TCP congestion research. Traffic can now be managed in a  \textit{centralised} fashion based on \textit{global knowledge} of the entire topology and traffic patterns.
A new line of centralised schedulers has emerged that  can achieve close to optimal bandwidth utilization.~\cite{hedera, fastpass, microte, b4, dionysus}
However, these schedulers are still  \textit{reactive}  in nature. The central controller responds to changes in the network or requests by applications, which may cost valuable roundtrip latency. Often, short-term flows or bursts are unaccounted for, which causes undesirable packet loss and backpropagating congestion. 

A much more desirable solution is a global, centralised arbiter which is able to predict and fairly distribute flows in the network before bursts or congestion occurs. By treating the network’s compute and forwarding power as a single finite resource, a controller could act like the OS scheduler distributing CPU time slices to processes. This design approach also follows SDN’s aspiration of introducing operating systems abstractions to the networking domain space.



In this project, we plan to explore the possibilities of a centralised, proactive flow scheduler. We ask ourselves the following research questions:
\begin{enumerate}
\item Is it possible to design a centralised token-based scheduling network?
\item Is it possible to predict traffic and preemptively schedule flows and token distribution in a datacenter context?
\item Using this approach, are we able to achieve better performance and utilization than existing solutions?
\end{enumerate}

In the scope of this course, we attempt to answer question 1 and design a simple token-based scheduler in Mininet. If we succeed, we will benchmark our results and evaluate the level of utilization compared to contemporary scheduling systems.



%
%%%%%%%%%%%%%%%%%%%%%%%%%%%%%%%%%%%%%%%%%%%%%%%%%%%%%%%
\section{Related Work}
\label{sec:related}
Nulla facilisi. Donec congue, quam sed faucibus varius, libero dolor mollis ipsum, vel placerat felis urna quis nibh. Morbi tempor ipsum nec urna porttitor interdum. Phasellus cursus nisi nec tempor blandit. Curabitur at sem convallis, sagittis purus ornare, sagittis urna. Nullam eget eleifend orci. Suspendisse vel justo at elit mollis porta. Nulla congue sapien nibh, tincidunt pulvinar orci elementum in. Aliquam mollis tempus velit non pretium. Aliquam hendrerit at orci id bibendum. Praesent gravida sapien eu eleifend faucibus.

Sed at tristique augue. Proin non venenatis lacus. Sed sodales ornare urna, id vulputate dui dictum in. Duis blandit, ex in ornare vulputate, est nibh ornare felis, vitae egestas libero leo nec enim. Fusce non arcu id eros suscipit feugiat. Nullam dapibus nibh a vestibulum maximus. Fusce convallis tortor odio, vitae egestas quam lacinia vitae. Aliquam laoreet convallis eros vel ultricies. Morbi non orci nunc. Proin accumsan posuere felis vitae pharetra. Aenean suscipit congue felis, in sollicitudin metus aliquam et. Pellentesque at commodo mauris. Morbi aliquet venenatis nisl, a cursus ligula cursus et. Cras sit amet lorem ac turpis vestibulum ultrices. Praesent cursus nibh vel malesuada convallis. 
%%%%%%%%%%%%%%%%%%%%%%%%%%%%%%%%%%%%%%%%%%%%%%%%%%%%%%%
\section{Design}
\label{sec:design}

Simple System
In our initial simple TCPToken design, a centralized controller regulates all node traffic by provisioning end-hosts with tokens.
These tokens act as the “currency” of the system. The total amount is fixed to the aggregate bandwidth available in the network and the traffic window size of sending end hosts is calculated based on the current token availability. If a node or switch is overburdened, it may notify the scheduler, which will adjust the traffic window of responsible sending nodes remotely. Any management or control operations such as token distribution and updates are priority-queued to guarantee a fast and low-latency response. In the case of an end-host requiring more bandwidth it can also notify the controller, which may comply based on priority and availability.
Initially, the controller will compute optimal route configuration based on the topology and link bandwidth using a simple heuristic bin-packing approach. End-hosts will be initialized with a fixed low-to-medium bandwidth guarantee, which will be adjusted over time.

Advanced System
The aforementioned system is intended only as a proof of concept. It is still a  synchronous and reactive approach, dependent on notifications and requests by overburdened switches and underprovisioned hosts. 
Ideally, all management in the TCPToken system should be asynchronous and preemptive. 
1) Paying for traffic
In a complete design, tokens will be used as a transport pass instead of simply acting as traffic shaping parameters. On a per-flow basis, nodes will be able to attach tokens to traffic as a form of payment. Enforcement is performed on the switch level, any flow that is not “paid” by a token will be dropped. Nodes may be handed different types of tokens to serve load-balancing and traffic shaping purposes. Applications operating on traditionally short-lived flows and bursts may sign their traffic with a different token type than long-lived permanent processes. Switches will forward traffic and balance flows accordingly with only minor interference from the central controller.

2) Predicting traffic
ML-based scheduling is an emerging research field, also known as Knowledge-Defined Networking. Datacenter traffic follows patterns, which can be predicted using statistical methods.
Combined with using tokens as a traffic engineering enforcement, a preemptive scheduling strategy in datacenters may be feasible. 

We do not intend to implement the advanced system for this project, but are using it as a guideline to motivate our research. Based on the experiences and knowledge gained from implementing the basic system, we will reevaluate the feasibility and practicality of the second concept.



Fault-Tolerance and Scalability
To handle fault-tolerance we plan to integrate a fallback-mechanism to conventional TCP, if the node has not received a TCPToken in a prefixed amount of time. End-hosts will only be marginally affected by a controller or link failure, as they are guaranteed a fixed amount of bandwidth based on their token availability.
To avoid overburdening the controller, we envision a distributed hive of schedulers each managing their own region in the final design. Add ONOS CITATION In case of failure, a shadow node or neighbouring device may take over computation temporarily, until the main node has recovered. CITATION
% %%%%%%%%%%%%%%%%%%%%%%%%%%%%%%%%%%%%%%%%%%%%%%%%%%%%%%%
\section{Motivation}
\label{sec:motivation}
Sed venenatis odio a augue porttitor, vel blandit tortor luctus. Pellentesque condimentum ante lacus, imperdiet commodo odio mattis a. Donec non consectetur libero. Donec efficitur erat nec dui convallis, sed efficitur nulla pharetra. Nullam sed feugiat est. Vestibulum est nisi, venenatis ac augue molestie, malesuada dignissim dolor. Aenean nisl ante, accumsan nec sagittis quis, euismod eleifend purus. Duis laoreet neque eros, non convallis nunc viverra eget. Sed eleifend neque neque, non faucibus justo vulputate vel. Fusce aliquam ultricies ligula ut efficitur. Nunc ut neque tristique, posuere sem vitae, consectetur nisl.

Donec pretium ut enim ac congue. Suspendisse nisl est, malesuada sit amet posuere a, gravida vitae lacus. Suspendisse et risus mauris. Nam imperdiet sit amet augue sed sodales. Quisque ante orci, aliquet non interdum nec, gravida sed enim. Nunc interdum, purus et scelerisque malesuada, diam lorem efficitur orci, et consequat ligula augue sed lacus. Morbi malesuada eleifend diam eu rhoncus. Class aptent taciti sociosqu ad litora torquent per conubia nostra, per inceptos himenaeos. 
%

%%%%%%%%%%%%%%%%%%%%%%%%%%%%%%%%%%%%%%%%%%%%%%%%%%%%%%
\section{Implementation}
\label{sec:implementation}
We intend to emulate our system in Mininet~\cite{mininet} to observe traffic 
patterns and infer a suitable token algorithm. Mininet has proven itself to be 
a viable tool to model new congestion control 
algorithms~\cite{mininet_learning}, and will help us prototype our concept 
efficiently. We will build a custom SDN controller that interacts with 
traditional OpenFlow software switches as well as end-hosts. End-hosts will run 
a custom real-world traffic generation script which adjusts based on 
information packets sent by the controller.
If time permits, we may expand our implementation to MaxiNet, which can emulate 
large-scale network stress tests on multiple physical hosts.

We initially considered a second implementation alternative in C/C++ based on 
the FastPass~\cite{fastpass} source code. This would provide us with a fully 
deployable system which we could fork our implementation from. A major 
advantage of this approach is the ability to test scenarios and traffic 
algorithms using real software code.
However, several concerns made us favour a Mininet emulation instead.
Firstly, FastPass relies on DPDK integration, which requires actual hardware 
interfaces. The central arbiter in the FastPass design would need to run on a 
dedicated machine, which increases prototyping and development complexity 
substantially.

Secondly, the FastPass code is highly specialized and optimized research code 
with only little available documentation. Modifying and evolving the source 
code will require thorough understanding of kernel and networking development, 
a significant time-sink. For a class project, these may be major initial 
hurdles, taking away from the research aspect of the design concept.
Consequently, we have decided to pursue an approach which allows us to quickly 
develop an understanding of the problem without being obstructed by engineering 
work.


%
%%%%%%%%%%%%%%%%%%%%%%%%%%%%%%%%%%%%%%%%%%%%%%%%%%%%%%
\section{Evaluation}
\label{sec:evaluation}

In the absence of a commercial data center, the implementation of our network 
design is going to be done on top of mininet with simulated FatTree topologies 
of varying size. We stress test with iPerf and simulate data center traffic 
using tcpreplay and packet traces provided by~\cite{traffic} .

To evaluate the general effectiveness of our system, we plan to measure against 
existing centralized as well as decentralized solutions.
The centralized design will be based on Hedera~\cite{hedera}, a common and 
influential datacenter scheduler. The decentralized congestion control 
mechanism will be DCTCP~\cite{dctcp}, a state-of-the-art TCP congestion 
algorithm.

Since we are primarily concerned about reducing the latency and packet drops 
while keeping utilisation at maximum, the measurements to get a good insight 
into how well the design can perform are as follows:
\begin{enumerate}

\item Latency: We are aiming for a low latency network which means that 99th 
percentile latency in the network across all flows should as low as possible. 
Latency should be measured when all hosts are sending packets at the maximum 
limit and also with random traffic patterns. There should be low latency even 
during sudden bursts as the transmission rate is limited by a base value.

\item Packet drop rate: Ideally this metric will approach zero, as the 
objective of Iroko is to minimize packet loss.

\item Fairness: Fairness can be tested by introducing a new host to a 
completely saturated network and increasing the transmission rate to see if all 
the hosts gets fair share of the total bandwidth. We are assuming all flows 
should be at equal priority. Differential priority is out of scope.

\item Responsiveness: This metric depends on the predictive power and 
efficiency of Iroko. It may be interesting to analyze the response time of the 
central scheduler compared to decentralized DCTCP and TCP. In addition, it is 
valuable to identify at what data center and flow size the central computing 
node of the network may become a bottleneck.

\item Network utilization: We will measure the used bandwidth and the total 
load on the hosts. These metrics can be measured by varying the load from each 
host. Load to utilization ratio should be 1 until the saturation point and 
after that utilization should stay stable at the maximum bisection bandwidth.
It is also important to check that there is no starvation happening in any 
host. This can be measured by simulating random traffic patterns and plotting 
residual bandwidth and excess load. We expect overall utilization to be lower 
compared to Hedera and DCTCP. Although Iroko will maximize goodput, we do not 
measure it on grounds of simplicity.

\end{enumerate}

As baseline we will also compare to the default TCP congestion algorithm to 
estimate how our scheduler fares against the default case. As our initial 
simple design is very constrained in its ability to route and control traffic 
we expect to achieve less overall optimality against DCTCP and Hedera, but 
still gain a substantial advantage over common TCP.

%%%%%%%%%%%%%%%%%%%%%%%%%%%%%%%%%%%%%%%%%%%%%%%%%%%%%%%
\section{Timeline}
\label{sec:timeline}
\begin{itemize}
\item \underline{October 15th:} Finalize draft and related work / background section.
\item \underline{ October 22nd:} Iterate over literature and thoroughly analyze related work. Set up an initial prototyping environment in Mininet on which each team member can work on.
\item \underline{October 30th:} Finalize the initial simple design and explore naive congestion modelling in Mininet. Integrate existing Mininet emulations of Hedera and DCTCP.
\item \underline{November 12th:} Develop initial naive TCPTokens prototype involving controller and end-host design.
\item \underline{November 19th:} Develop intelligent centralised congestion control heuristic.
\item \underline{November 26th:} Run evaluations and microbenchmarks.
\item \underline{December 3rd:} Iterate and improve the scheduler, explore topology and bandwidth constraints. Test at scale.
\item \underline{December 10th:} Write report and prepare presentation.
\end{itemize}

%%%%%%%%%%%%%%%%%%%%%%%%%%%%%%%%%%%%%%%%%%%%%%%%%%%%%%
\section{Introduction}
\label{sec:intro}

Initially, research in network congestion was dominated by the assumption of a decentralised, autonomous network - end-hosts only have control over the amount of traffic they send, and are unaware of the intentions or traffic rates of their peers. Similarly, switches and routers are unaware of global traffic patterns and only forward based on their local notion of optimality.

In line with these assumptions, TCP was designed to optimize traffic globally on a simple fairness principle: nodes react to packet loss and assume that others will behave accordingly. An equilibrium is reached when all end-hosts achieve a traffic rate that, in sum, conforms to the maximum available bandwidth of the most congested link.
TCP works well  in scenarios where many different distrustful participants compete for limited bandwidth. Still, TCP is a \textit{reactive protocol}; the fact that packet loss and latency increases occur in the network already indicates a problem. Packet loss is primarily due to overflowing queues in forwarding elements, implying that traffic has not been optimally distributed. Ideally, a network should always be "zero-queue", i.e., latency will merely be induced by propagation, and not queuing delay. 

Queueing has generally not been a dominating issue in wide-area and enterprise networks, as traffic is sufficiently distributed and diverse, with only few "hot" target hosts.~\cite{hedera, microte} Traffic optimization is a substantial challenge; network operators have no control over the individual network elements nor its participants. Under these conditions, TCP and its extensions can be considered a best-effort solution.

Developments in the past decade have changed the general networking environment. Datacenters have emerged as an exciting new research frontier, posing novel design challenges and opportunities.

Driven by minimization of costs and maximization of compute power, data centers must  run at maximum utilization to achieve an optimal compute/cost ratio. Inefficient routing can quickly lead to bufferbloat~\cite{bufferbloat} and the eventual collapse of a high-load network, requiring more sophisticated approaches to solve congestion control. 
On the other hand, operators now have the ability to freely control and adapt their network architecture, leading to highly customized systems and fine-grained optimization. As a result , Software-Defined Networking (SDN) has emerged as a new networking paradigm. Moving away from the principle of distributed communication and routing, SDN introduces the notion of "centralized management". A single controller with global knowledge is able to automatically modify and adapt the forwarding tables of all switches in the network, while notifying end hosts of changes in the network.
These two new trends in system design facilitated impactful new innovation opportunities in the space of TCP congestion research. Traffic can now be managed in a  \textit{centralised} fashion based on \textit{global knowledge} of the entire topology and traffic patterns.

A new line of centralised schedulers has emerged that  can achieve close to optimal bandwidth utilization.~\cite{hedera, fastpass, microte, b4, dionysus}
However, these schedulers are still  \textit{reactive}  in nature. The central controller responds to changes in the network or requests by applications, which may cost valuable roundtrip latency. Often, short-term flows or bursts are unaccounted for, which causes undesirable packet loss and backpropagating congestion. 

A much more desirable solution is a global, centralised arbiter which is able to predict and fairly distribute flows in the network before bursts or congestion occurs. By treating the network’s compute and forwarding power as a single finite resource, a controller acts like the OS scheduler distributing CPU time slices to processes. This design approach follows SDN’s aspiration of introducing operating systems abstractions to the networking domain space.

In this project, we plan to explore the possibilities of a centralised, proactive flow scheduler. We ask ourselves the following research questions:
\begin{enumerate}
\item Is it possible to design a centralised token-based scheduling network?
\item Is it possible to predict traffic and preemptively schedule flows and token distribution in a datacenter context?
\item Using this approach, are we able to achieve better performance and utilization than existing solutions?
\end{enumerate}

In the scope of this course, we attempt to answer question one and design a simple token-based scheduler in Mininet. If successful, we will benchmark our results and evaluate the level of utilization compared to contemporary scheduling systems.

%%%%%%%%%%%%%%%%%%%%%%%%%%%%%%%%%%%%%%%%%%%%%%%%%%%%%%
\section{Related Work}
\label{sec:related}

Most of the work in congestion control relies on a centralised scheduler which periodically gets updates about the flows in the network and reserves specific paths for long running flows.

MicroTE~\cite{microte} does predictive OpenFlow routing and scheduling using a central SDN controller. Hedera~\cite{hedera} schedules large so called "elephant" flows by gathering switch statistics. However it pays poor attention to short term flows and network latency. FastPass~\cite{fastpass} on the other hand, proposes a zero-queue network and promises a very low latency network. The feasibility of this proposal remains questionable when considering the high cost of implementation.

Recent proposals on decentralised scheduling like ExpressPass~\cite{expresspass} have heavily influenced our research. ExpressPass posits a credit based system similar to TCPTokens in order to control the congestion in the network. Effecting flow rerouting using ML techniques in KDN~\cite{kdn} and other works shows promising signs of further improvements in this area.

%%%%%%%%%%%%%%%%%%%%%%%%%%%%%%%%%%%%%%%%%%%%%%%%%%%%%%
\section{Design / Proposed Approach}
\label{sec:design}

\subsection{Simple System}
In our initial simple TCPToken design, a centralized controller regulates all node traffic by provisioning end-hosts with tokens.
These tokens act as the "currency" of the system. The total amount is fixed to some factor of the aggregate bandwidth available in the network, and the traffic window size of sending end hosts is calculated based on the host’s current token availability. If a node or switch is overburdened, it may notify the scheduler, which will adjust the traffic window of responsible sending nodes remotely. Management and control operations including token distribution and updates are priority-queued to guarantee a fast and low-latency response. If an end-host requires more bandwidth, it can also notify the controller, which may comply based on priority and availability.
Initially, the controller will compute optimal route configuration based on the topology and link bandwidth using a simple heuristic bin-packing approach. End-hosts will be initialized with a fixed low-to-medium bandwidth guarantee that will be adjusted over time.

\subsection{Advanced System}
The aforementioned system is intended only as a proof of concept. It is still a  synchronous and reactive approach, dependent on notifications and requests by overburdened switches and underprovisioned hosts. 
Ideally, all management in the TCPToken system should be asynchronous and preemptive. 
\subsubsection{ Paying for traffic}
In a complete design, tokens will be used as a transport pass instead of simply acting as traffic shaping parameters. On a per-flow basis, nodes will be able to attach tokens to traffic as a form of payment. Enforcement is performed on the switch level, any flow that is not "paid" by a token will be dropped. Nodes may be handed different types of tokens to serve load-balancing and traffic shaping purposes. Applications operating on traditionally short-lived flows and bursts may sign their traffic with a different token type than long-lived permanent processes. Switches will forward traffic and balance flows accordingly with only minor interference from the central controller.

\subsubsection{Predicting traffic}
ML-based scheduling is an emerging research field, also known as Knowledge-Defined Networking. Datacenter traffic follows patterns, which can be predicted using statistical methods.
Combined with using tokens as a traffic engineering enforcement, a preemptive scheduling strategy in datacenters may be feasible. 

We do not intend to implement the advanced system for this project, but are using it as a guideline to motivate our research. Based on the experiences and knowledge gained from implementing the basic system, we will reevaluate the feasibility and practicality of the second concept.



\subsubsection{Fault-Tolerance and Scalability}
To handle fault-tolerance we plan to integrate a fallback-mechanism to conventional TCP, if the node has not received a TCPToken in a prefixed amount of time. End-hosts will only be marginally affected by a controller or link failure, as they are guaranteed a fixed amount of bandwidth based on their token availability.

To avoid overburdening the controller, we envision a distributed hive of schedulers (e.g., Onos) each managing their own region in the final design.~\cite{onos} In case of failure, a shadow node or neighbouring device may take over computation temporarily, until the main node has recovered.~\cite{disco}





%%%%%%%%%%%%%%%%%%%%%%%%%%%%%%%%%%%%%%%%%%%%%%%%%%%%%%
\section{Implementation}
\label{sec:implementation}
We intend to emulate our system in Mininet~\cite{mininet} to observe traffic patterns and infer a suitable token algorithm. Mininet has proven itself to be a viable tool to model new congestion control algorithms~\cite{mininet_learning}, and will help us prototype our concept efficiently. We will build a custom SDN controller that interacts with traditional OpenFlow software switches as well as end-hosts. End-hosts will run a custom real-world traffic generation script which adjusts based on information packets sent by the controller. 
If time permits, we may expand our implementation to MaxiNet, which can emulate large-scale network stress tests of thousands of nodes on multiple physical hosts.

We initially considered a second implementation alternative in C/C++ based on the FastPass~\cite{fastpass} source code. This would provide us with a fully deployable system which we could fork our implementation from. A major advantage of this approach is the ability to test scenarios and traffic algorithms using real software code. 
However, several concerns made us favour a Mininet emulation instead.
Firstly, FastPass relies on DPDK integration, which requires actual hardware interfaces. The central arbiter in the FastPass design would need to run on a dedicated machine, which increases prototyping and development complexity substantially.

Secondly, the FastPass code is highly specialized and optimized research code with only little available documentation. Modifying and evolving the source code will require thorough understanding of kernel and networking development, a significant time-sink. For a class project, these may be major initial hurdles, taking away from the research aspect of the design concept.
Consequently, we have decided to pursue an approach which allows us to quickly develop an understanding of the problem without being obstructed by engineering work.








%%%%%%%%%%%%%%%%%%%%%%%%%%%%%%%%%%%%%%%%%%%%%%%%%%%%%%
\section{Evaluation}
\label{sec:evaluation}
To evaluate the effectiveness of our system, we plan to measure against existing centralized as well as decentralized solutions. We will run datacenter traffic on a fat-tree topology of various depth and observe the effect each congestion algorithm will have on the overall bandwidth utilization and latency.
The centralized design will be based on Hedera~\cite{hedera}, a common and influential datacenter scheduler.

The decentralized congestion control mechanism will be DCTCP~\cite{dctcp}, a state-of-the-art TCP congestion algorithm. As baseline we will also compare to the default TCP congestion algorithm to estimate how our scheduler fares against the default case. As our initial simple design is very constrained in its ability to route and control traffic we expect to achieve less optimality against DCTCP and Hedera, but gain a substantial advantage over common TCP.

%%%%%%%%%%%%%%%%%%%%%%%%%%%%%%%%%%%%%%%%%%%%%%%%%%%%%%
\section{Timeline}
\label{sec:timeline}
\begin{itemize}
\item \underline{October 15th:} Finalize draft and related work / background section.
\item \underline{ October 22nd:} Iterate over literature and thoroughly analyze related work. Set up an initial prototyping environment in Mininet on which each team member can work on.
\item \underline{October 30th:} Finalize the initial simple design and explore naive congestion modelling in Mininet. Integrate existing Mininet emulations of Hedera and DCTCP.
\item \underline{November 12th:} Develop initial naive TCPTokens prototype involving controller and end-host design.
\item \underline{November 19th:} Develop intelligent centralised congestion control heuristic.
\item \underline{November 26th:} Run evaluations and microbenchmarks.
\item \underline{December 3rd:} Iterate and improve the scheduler, explore topology and bandwidth constraints. Test at scale.
\item \underline{December 10th:} Write report and prepare presentation.
\end{itemize}



\bibliographystyle{ACM-Reference-Format}
\bibliography{paper} 

\end{document}
